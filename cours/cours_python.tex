\documentclass[a4paper,12pt]{article}

% Packages de base
\usepackage[utf8]{inputenc}
\usepackage[T1]{fontenc}
\usepackage[english,french]{babel}
\usepackage[margin=2.5cm]{geometry}

% Packages utiles
\usepackage{graphicx} % Pour les images
\usepackage{hyperref} % Liens cliquables
\usepackage{amsmath}  % Mathématiques avancées
\usepackage{enumitem} % Listes personnalisées
\usepackage{titling}  % Permet de mofif le title

% Premiere page
\title{Python pour les incultes}
\author{Bodin Jules}
\date{\today}

\renewcommand{\maketitlehookd}{%
    \begin{center}
    \includegraphics[keepaspectratio=true, width=0.8\textwidth]{../images/Logo_Python.jpg}
    \end{center}
}

\begin{document}

% Page de titre
\maketitle

% Page d'introduction avant la table des matieres
% Expliques comment est design ce pdf

\newpage

Ce document a pour but d'enseigner les rudiments de Python. 

\newpage

\tableofcontents % Table des matières optionnelle

\newpage

\section{Introduction}
Votre introduction ici.

\end{document}
